\documentclass{article}
\usepackage[utf8]{inputenc}
\usepackage{comment} % enables the use of multi-line comments (\ifx \fi) 
\usepackage{fullpage} % changes the margin
\usepackage{hyperref}
\usepackage{amsmath}
\usepackage{mathtools}
\usepackage{booktabs} % For formal tables
\usepackage{listings}
\begin{document}
\noindent
\large\textbf{Software Engineering Process: Problem 2} \hfill \textbf{Amit Sachdeva} 

\normalsize Topic: Functional Requirements of Gamma Function \hfill \textbf{40084627} 

Prof. P. Kamthan \hfill Due Date: 12/July/2019

\section{Introduction} 
\textbf{Gamma Function: } It is commonly referred as factorial function for complex numbers.\\  It is derived by Daniel Bernoulli.

\section{Overall Description} 
This is a project based on gamma function in which we are making calculator for gamma value. User can insert any real value and expect real value except on boundary conditions.

\section{Stakeholders} 
\textbf{Users 1:} This function is mostly used in physics calculations. So, most important stakeholders are scientists for their calculations.
\newline
\textbf{Users 2:} This function is also used in basic maths calculations or any analytically field. 



\section{Formulas Related to Function}
\begin{itemize}
\item $  \Gamma \left( x \right) = \int\limits_0^\infty {s^{x - 1} e^{ - s} ds} \enspace \forall \enspace Re(x)>0$
\item $  \Gamma \left( 1/2 \right) = \sqrt{\pi}$
\item $  n! = n*(n-1)!$
\item $  \Gamma \left( x \right) = x\Gamma \left( x-1\right) $
\item $  \Gamma \left( 0 \right) = undefined $
\end{itemize}

\section{Domain of Function}
$\forall$ Real numbers excluding all negative integers \\
$(0, \infty)$

\section{Co domain of Function}
\begin{itemize}
\item It ranges from $(1, \infty)$
\item For positive integers, we returns integer value as normal factorial
\item For other real numbers, we use integral function.
\end{itemize}

\section{Input/Output of Function}
\begin{itemize}
\item For \textbf{negative input} $\forall x<0 $, \textbf{Function} will return \textbf{undefined} response
\item For \textbf{$x=0$}, \textbf{Function} will return \textbf{undefined}
\item For \textbf{$Re(x) > 0$}, \textbf{Function} will return positive real value
\end{itemize}

\section{Risk/Constraints of Gamma Function}
We cannot \textbf{input} value of \textbf{non negative integers} as well as negative real number
\end{document}
