\documentclass{article}
\usepackage[utf8]{inputenc}
\usepackage{natbib}
\usepackage{graphicx}
\usepackage[utf8]{inputenc}
\usepackage{comment} % enables the use of multi-line comments (\ifx \fi) 
\usepackage{fullpage} % changes the margin
\usepackage{hyperref}
\usepackage{amsmath}
\usepackage{mathtools}
\usepackage{booktabs} % For formal tables
\usepackage{listings}
\usepackage{tabto}
\begin{document}
\noindent
\large\textbf{Software Engineering Process} \hfill \textbf{Amit Sachdeva} 

\normalsize Topic: Final Deliverables \hfill \textbf{40084627} 

Prof. P. Kamthan \hfill Due Date: 19/July/2019

\begin{center}
    \section*{Gamma Function}
    \section*{Problem 1}
\end{center}
\section{Introduction} 
\textbf{Gamma Function: } It is commonly referred as factorial function for complex numbers. It is derived by Daniel Bernoulli. The gamma function $\gamma(z)$ is defined for all complex values of z larger than zero. Complex number can be consist of real and imaginary number, like $z = a + i b$ in which a and b can real numbers. A complex number is typically written in the form where sigma a is the real part and it is the imaginary part.
\begin{figure}[h!]
\centering
\includegraphics[scale=0.4]{gamma1}
\caption{Gamma Function}
\label{fig:Gamma Function}
\end{figure}
\section{Overall Description} 
This is a project based on gamma function in which we are making calculator for gamma value. User can insert any real value and expect real value except on boundary conditions.

\section{Stakeholders} 
\textbf{Users 1:} This function is mostly used in physics calculations. So, most important stakeholders are scientists for their calculations.
\newline \newline
\textbf{Users 2:} This function is also used in basic maths calculations or any analytically field. 



\section{Related to Function}
\subsection{Formulas}
\begin{itemize}
\item \textbf{Formula1: } $  \Gamma \left( x \right) = \int\limits_0^\infty {s^{x - 1} e^{ - s} ds} \enspace \forall \enspace Re(x)>0$
\item \textbf{Formula2: } $  \Gamma \left( 1/2 \right) = \sqrt{\pi}$
\item \textbf{Formula3: } $  n! = n*(n-1)!$
\item \textbf{Formula4: } $  \Gamma \left( x \right) = x\Gamma \left( x-1\right) $
\item \textbf{Formula5: } $  \Gamma \left( 0 \right) = undefined $
\end{itemize}
\subsection{Popular Constant Values of Function}
\begin{itemize}
\item \textbf{Constant 1: } $  \Gamma \left( 0 \right) = undefined $
\item \textbf{Constant 2: } $  \Gamma \left( 1 \right) = 1 $
\item \textbf{Constant 3: } $  \Gamma \left( 2 \right) = 1 $
\item \textbf{Constant 4: } $  \Gamma \left( 3 \right) = 6 $
\item \textbf{Constant 5: } $  \Gamma \left( 3/2 \right) = 0.886 $
\item \textbf{Constant 6: } $  \Gamma \left( -3/2 \right) = 2.36 $
\item \textbf{Constant 7: } $  \Gamma \left( -1/2 \right) = -3.54 $
\end{itemize}
\subsection{Domain of Function}
$\forall$ Real numbers excluding negative values \\
$[0, \infty)$

\subsection{Co domain of Function}
\begin{itemize}
\item It ranges from $(0, \infty)$
\item For positive integers, we returns integer value as normal factorial
\item For other real numbers, we use integral function.
\end{itemize}
\begin{center}
    \section*{Problem 2}
\end{center}
\section{Requirements/Constraints of Function}
\subsection{Requirements}
\begin{itemize}
\item \textbf{Req1}: For \textbf{Large input} in positive value, it will return infinity as \textbf{Const3}.
\item \textbf{Req2}: For \textbf{negative input} $\forall x<0 $, \textbf{Function} will return \textbf{input error}, keeping in mind \textbf{Const1} and \textbf{Const2}
\item \textbf{Req3}: For \textbf{$x=0$}, \textbf{Function} will return \textbf{undefined}, keeping in mind \textbf{Const1}
\item \textbf{Req4}: For \textbf{$Re(x) > 0$}, \textbf{Function} will return positive real value, keeping in mind \textbf{Const1}
\end{itemize}

\subsection{Constraints}
\begin{itemize}
\item \textbf{Constraint 1}: For Input, types must be Integer, Double, Float data types
\item \textbf{Constraint 2}: We cannot \textbf{input} value of \textbf{non negative values}
\item \textbf{Constraint 3}: We cannot input the value large positive number as it will return infinity as a programming language constraint
\end{itemize}

\begin{center}
    \section*{Problem 3}
\end{center}

\section{Algorithms}
\subsection{Pseudo Code 1}
\subsubsection{Algorithm}
This algorithm is based on calculating
1) function yAxisValue(Argument x, Argument s) \{ \newline
2) \hspace{15pt} Calculate the value using $value = {s^{x - 1} e^{ - s}}$\newline
3) \hspace{15pt} return value\newline
4) end \newline
5) \}  \newline
6) function gammaFunction(Argument x) \{ \newline
7) \hspace{15pt} if x $<$ 0\newline
8) \hspace{35pt} then raise Input Error\newline
9) \hspace{15pt} if x $>$ 170\newline
10) \hspace{35pt} then return "Infinity"\newline
11) \hspace{15pt}Initialize finalData with 0\newline
12) \hspace{15pt}Set Interval for gap = $10 ^ {-3}$ \newline
13) \hspace{15pt}while loop i for range(0,Infinity)\newline
14) \hspace{35pt}Add the finalData by using formula of trapezium using \newline 
15) \hspace{35pt} \textbf{$1/2*gap*(yAxisValue(i) + yAxisValue(i-gap))$} \newline
16) \hspace{35pt} increment i with gap value \newline
17) \hspace{15pt} return finalData\newline
18) end \newline
19) \} \newline
20) \{ \newline
21) In main function \newline
22) \hspace{15pt} Take a input of x \newline
23) \hspace{15pt} Call gammaFunction with input x-1 as a argument \newline
24) \}
\subsubsection{Advantages}
\begin{itemize}
\item Get More Precise Values for input as tested with existing results
\item Using basic core approach of integration
\end{itemize}
\subsubsection{Disadvantages}
\begin{itemize}
\item We are iterating the loop at a large value so it takes time
\end{itemize}
\subsection{Pseudo Code 2}
\subsubsection{Algorithm}
\subsubsection{Advantages}
\begin{itemize}
\item We are using constant pre-calculated value which increase the speed of algorithm
\item This algorithm has less calculation which reduce complexity of code
\end{itemize}
\subsubsection{Disadvantages}
\begin{itemize}
\item We have generated constant value which not always give accurate value
\end{itemize}


\section{References}
\begin{itemize}
\item https://www.ncbi.nlm.nih.gov/pmc/articles/PMC4247832/
\item https://medium.com/cantors-paradise/the-riemann-hypothesis-explained-fa01c1f75d3f
\end{itemize}
\end{document}
